\PassOptionsToPackage{unicode=true}{hyperref} % options for packages loaded elsewhere
\PassOptionsToPackage{hyphens}{url}
%
\documentclass[
  12pt,
]{article}
\usepackage{lmodern}
\usepackage{amssymb,amsmath}
\usepackage{ifxetex,ifluatex}
\ifnum 0\ifxetex 1\fi\ifluatex 1\fi=0 % if pdftex
  \usepackage[T1]{fontenc}
  \usepackage[utf8]{inputenc}
  \usepackage{textcomp} % provides euro and other symbols
\else % if luatex or xelatex
  \usepackage{unicode-math}
  \defaultfontfeatures{Scale=MatchLowercase}
  \defaultfontfeatures[\rmfamily]{Ligatures=TeX,Scale=1}
\fi
% use upquote if available, for straight quotes in verbatim environments
\IfFileExists{upquote.sty}{\usepackage{upquote}}{}
\IfFileExists{microtype.sty}{% use microtype if available
  \usepackage[]{microtype}
  \UseMicrotypeSet[protrusion]{basicmath} % disable protrusion for tt fonts
}{}
\makeatletter
\@ifundefined{KOMAClassName}{% if non-KOMA class
  \IfFileExists{parskip.sty}{%
    \usepackage{parskip}
  }{% else
    \setlength{\parindent}{0pt}
    \setlength{\parskip}{6pt plus 2pt minus 1pt}}
}{% if KOMA class
  \KOMAoptions{parskip=half}}
\makeatother
\usepackage{xcolor}
\IfFileExists{xurl.sty}{\usepackage{xurl}}{} % add URL line breaks if available
\IfFileExists{bookmark.sty}{\usepackage{bookmark}}{\usepackage{hyperref}}
\hypersetup{
  pdftitle={Correction},
  pdfauthor={Alexis Lignoux},
  pdfborder={0 0 0},
  breaklinks=true}
\urlstyle{same}  % don't use monospace font for urls
\usepackage{graphicx,grffile}
\makeatletter
\def\maxwidth{\ifdim\Gin@nat@width>\linewidth\linewidth\else\Gin@nat@width\fi}
\def\maxheight{\ifdim\Gin@nat@height>\textheight\textheight\else\Gin@nat@height\fi}
\makeatother
% Scale images if necessary, so that they will not overflow the page
% margins by default, and it is still possible to overwrite the defaults
% using explicit options in \includegraphics[width, height, ...]{}
\setkeys{Gin}{width=\maxwidth,height=\maxheight,keepaspectratio}
\setlength{\emergencystretch}{3em}  % prevent overfull lines
\providecommand{\tightlist}{%
  \setlength{\itemsep}{0pt}\setlength{\parskip}{0pt}}
\setcounter{secnumdepth}{5}

% set default figure placement to htbp
\makeatletter
\def\fps@figure{htbp}
\makeatother

\AtBeginDocument{\let\maketitle\relax}
%\usepackage{soul} % Pour rayer
\usepackage[normalem]{ulem} % Pour souligner
\usepackage{times}
\usepackage[T1]{fontenc}
\usepackage{textcomp}
\usepackage[utf8]{inputenc}
\usepackage{hyperref}
\usepackage[a4paper, left=2cm, right=2cm, top = 2cm, bottom = 2cm]{geometry}
\usepackage{float}
\usepackage{multicol}
\usepackage{multirow}
\usepackage{array}
\usepackage{calc}
\usepackage{pdflscape}
\usepackage{ragged2e}
\usepackage{soulutf8}
\usepackage{xspace}
\usepackage{enumitem}
\usepackage{tkz-tab}
\usepackage{sectsty}
\usepackage{pgfplots}

\pgfplotsset{width=14cm,compat=1.9}

\setlength\parindent{0pt}
\setlength\headheight{20pt}

\title{Correction}
\author{Alexis Lignoux}
\date{\today}

\begin{document}
\maketitle

\begin{center}

{\LARGE Exercice corrigé}

\end{center}

\vspace{1.5cm}

\normalsize

\textit{Déterminer le tableau de variation de la fonction suivante :}

\begin{center}
\large
$f(x)=-8x^3+3x^2+9x+4$
\end{center}
\vspace{1.2cm}
\normalsize

On commence par déterminer la dérivée de notre fonction. Dans notre cas
la dérivée s'écrit :

\begin{center}

        \large
$f'(x)=-24x^2+6x+9$
\end{center}
\bigskip
\normalsize

L'étape suivante consiste à déterminer quand la dérivée s'annule, c'est
à dire résoudre l'équation :

\begin{center}
\large
$f'(x)=0$

$\Leftrightarrow$  $-24x^2+6x+9=0$

\end{center}
\smallskip
\normalsize

Dans notre cas, \(f'(x)\) est un polynôme de degré \(2\) donc on calcule
le \(\Delta\) :

\large

\phantom{$\Rightarrow$} \(\Delta=b^2-4ac\)

\(\Leftrightarrow\) \(\Delta=(6)^2-4\times(-24)\times(9)\)

\(\Leftrightarrow\) \(\Delta=900\)

\(\Leftrightarrow\) \(\Delta>0\)

\smallskip

\normalsize

On a donc deux solutions à l'équation \(f'(x)=0\). Déterminons à présent
ces deux solutions :

\large

\phantom{$\Rightarrow$} \(x_1=\frac{-b-\sqrt{\Delta}}{2a}\)

\(\Leftrightarrow\) \(x_1=\frac{-(6)-\sqrt{900}}{2\times(-24)}\)

\(\Leftrightarrow\) \(x_1=0.75\)

\medskip

\phantom{$\Rightarrow$} \(x_2=\frac{-b+\sqrt{\Delta}}{2a}\)

\(\Leftrightarrow\) \(x_2=\frac{-(6)+\sqrt{900}}{2\times(-24)}\)

\(\Leftrightarrow\) \(x_2=-0.5\)

\normalsize

\bigskip

On calcule les valeurs prises par la fonction en \(x_1\) et \(x_2\) :

\large

\phantom{$\Leftrightarrow$} \(f(x_1)=-8x_1^3+3x_1^2+9x_1+4\)

\(\Leftrightarrow\) \(f(0.75)=-8(0.75)^3+3(0.75)^2+9(0.75)+4\)

\(\Leftrightarrow\) \(f(0.75)=9.0625\)

\medskip

\phantom{$\Leftrightarrow$} \(f(x_2)=-8x_2^3+3x_2^2+9x_2+4\)

\(\Leftrightarrow\) \(f(-0.5)=-8(-0.5)^3+3(-0.5)^2+9(-0.5)+4\)

\(\Leftrightarrow\) \(f(-0.5)=1.25\)

\normalsize

\vspace{1.2cm}

On obtient le tableau de variation suivant :

\medskip

\large

\begin{center}

\begin{tikzpicture}

\tkzTabInit{$x$ / 1.5 , $f'(x)$ / 1.5, $f(x)$ / 2.5}{$-\infty$,$-0.5$, $0.75$, $+\infty$}
\tkzTabLine{, -, z, +, z, -, }
\tkzTabVar{+/ $+\infty$, -/ $1.25$, +/ $9.062$, -/ $-\infty$}
\end{tikzpicture}

\medskip
\normalsize
Tableau de variation de la fonction $f(x)$ :

\end{center}

\vspace{1.5cm}

\newpage

\begin{center}
\begin{tikzpicture}
\begin{axis}[
axis lines = left,
xlabel = \(x\),
ylabel = {\(f(x)\)},
]
%Below the blue function is defined
\addplot [
domain=-1.33333333333333:1.58333333333333,
samples=100,
color=blue,
]
{-8*x^3+3*x^2+9*x+4};
\addlegendentry{\(-8x^3+3x^2+9x+4\)}
\end{axis}
\end{tikzpicture}
\medskip
\normalsize
Représentation graphique de la fonction $f(x)$ :

\end{center}

\end{document}
